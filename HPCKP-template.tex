\documentclass{beamer}
\setbeamertemplate{navigation symbols}{}
\usepackage{beamerthemeshadow}
\usepackage[english]{layout}
\usepackage[utf8]{inputenc}
\usepackage[english]{babel}
\usepackage[T1]{fontenc}
\usepackage{amsmath, soul, color, multicol, type1cm, verbatim, latexsym, float,listings}
\usepackage[official]{eurosym}
\usepackage{thumbpdf}
\setbeamercovered{transparent}
\usetheme{Frankfurt}
\usefonttheme{professionalfonts}
% Setup TikZ
\usepackage{tikz}
\usetikzlibrary{arrows}
\tikzstyle{block}=[draw opacity=0.7,line width=1.4cm]

\newcommand\BackgroundPicture[1]{%
\setbeamertemplate{background canvas}{%
\parbox[c][\paperheight]{\paperwidth}{%
\vfill \hfill \includegraphics[height=\paperheight,width=\paperwidth]{#1}
\hfill \vfill
}}}

%HPCKP Colors <---------------------------------------------------------------------------------------
\usecolortheme[RGB={112,204,10}]{structure}
\definecolor{hpcnowdark}{HTML}{2F4447}
\definecolor{hpcnowlight}{HTML}{CED9DF}
\setbeamercolor{block body}{fg=black}
\setbeamercolor{block body alerted}{fg=black}
\setbeamercolor{normal text}{fg=hpcnowlight}
\frenchspacing
\setbeamertemplate{blocks}[default]%[shadow=false]
\useinnertheme{circles}
\setbeamertemplate{title page}[default][center,rounded=true,shadow=false]
%HPCKP Custom Code Hightlight <---------------------------------------------------------------------------------------
\lstdefinestyle{customcode}{
  belowcaptionskip=1\baselineskip,
  breaklines=true,
  xleftmargin=\parindent,
  showstringspaces=false,
  basicstyle=\ttfamily,
  keywordstyle=\bfseries\color{green!40!black},
  commentstyle=\itshape\color{purple!40!black},
  identifierstyle=\color{blue},
  stringstyle=\color{orange},
}
%TITLE <---------------------------------------------------------------------------------------
\title{Title}
\subtitle{Subtitle}
\author{Author1 \& Author2 (Company)}
\date{}

\begin{document}
{
\setbeamertemplate{background canvas}{\includegraphics[height=\paperheight,width=\paperwidth]{hpckp_slide00.png}} 
\begin{frame}[plain]
\vspace*{+2.5cm}
  \titlepage
\end{frame}
}


\BackgroundPicture{hpckp_bg.jpg}
\begin{frame}
\frametitle{Agenda}
\begin{multicols}{2}
  \tableofcontents
\end{multicols}
\end{frame}

\section{Standard block}
\frame[t]
{
  \frametitle{About HPCKP}
  \begin{block}{HPC Knowledge Portal}
HPCKP (High-Performance Computing Knowledge Portal) is an Open Knowledge project focused on the technology transfer and the Knowledge sharing in the HPC Science field.
  \end{block}
}

\section{Lists}
\subsection{bullets}
% List with bullets (itemize) <---
% The [<+-| alert@+>] will create a new slide for each element with highliting 
\frame[t]
{
  \frametitle{About Something}
   \begin{block}{About Something}
   \begin{itemize}%[<+-| alert@+>]
	\item one
	\item two
	\item three
	\item four
   \end{itemize}
  \end{block}
}

\subsection{numbers}
% List with numbers (enumerate) <---
% The [<+-| alert@+>] will create a new slide for each element with highliting 
\frame[t]
{
  \frametitle{About Something}
   \begin{block}{About Something}
   \begin{enumerate}%[<+-| alert@+>]
	\item one
	\item two
	\item three
	\item four
   \end{enumerate}
  \end{block}
}

\section{Slides with code}
\subsection{verbatim}
% If you want to include some code (verbatim or listings) you will need to add "fragile" in the frame options <---
\begin{frame}[fragile]
  \frametitle{Submitting a Job}
      \begin{block}{Job Description Example : SMP}
\begin{small}
\begin{verbatim}
#!/bin/bash
#SBATCH -o /share/src/app/example/app-%j.%N.out
#SBATCH -D /share/src/app/example
#SBATCH -J MIGRATESMP
#SBATCH --get-user-env
#SBATCH --ntasks=1
#SBATCH --cpu_bind=core
#SBATCH --export=NONE
#SBATCH --time=08:00:00
ml load migrate/3.4.4-smp
export OMP_NUM_THREADS=4
srun migrate-n-thread parmfile.testml -nomenu
\end{verbatim}
\end{small}
  \end{block}
\end{frame}

\subsection{code highlight with listings}
% If you want to include some code (verbatim or listings) you will need to add "fragile" in the frame options <---
\begin{frame}[fragile]
  \frametitle{listings}
      \begin{block}{Code highlight}
\begin{tiny}
\lstset{language=Python,escapechar=@,style=customcode}
\begin{lstlisting} 
def pi_chudnovsky(one=1000000):
    k = 1
    a_k = one
    a_sum = one
    b_sum = 0
    C = 640320
    C3_OVER_24 = C**3 // 24
    while 1:
        a_k *= -(6*k-5)*(2*k-1)*(6*k-1)
        a_k //= k*k*k*C3_OVER_24
        a_sum += a_k
        b_sum += k * a_k
        k += 1
        if a_k == 0:
            break
    total = 13591409*a_sum + 545140134*b_sum
    pi = (426880*sqrt(10005*one, one)*one) // total
    return pi
\end{lstlisting}
\end{tiny}
  \end{block}
\end{frame}

\section{Multiple boxes}
% Example of three boxes  <---
\begin{frame}{Three boxes example}
  \begin{columns}[t]
    \column{.3\textwidth}
    \begin{block}{matrix one}
      $G\colon$
      \begin{tabular}{ccc}
        A & B & C \\\hline
        2 & 2 & 2 \\
        0 & 2 & 0 \\
        2 & 0 & 0 \\
        0 & 2 & 2 
      \end{tabular}
    \end{block}

    \column{.3\textwidth}
    \begin{alertblock}{matrix two}
      $H\colon$
      \begin{tabular}{ccc}
        A & B & C \\\hline
        1 & 0 & 0 \\
        0 & 1 & 1 \\
        0 & 0 & 0 \\
        0 & 1 & 0 \\
        0 & 0 & 0 \\
        1 & 0 & 0 \\
        0 & 0 & 0 \\
        0 & 1 & 1 
      \end{tabular}
    \end{alertblock}

    \column{.4\textwidth}
    \begin{block}{Figure}
      \begin{center}
        \begin{tikzpicture}[auto,thick]
          \tikzstyle{node}=%
          [%
            minimum size=10pt,%
            inner sep=0pt,%
            outer sep=0pt,%
            ball color=example text.fg,%
            circle%
          ]
        
          \node [node] {} [->]
            child {node [node] {} edge from parent node[swap]{A}}
            child {node [node] {}
              child {node [node] {} edge from parent node{C}}
              edge from parent node{B}
            };
        \end{tikzpicture}
      \end{center}
    \end{block}
  \end{columns}
\end{frame}

%Questions 
\section{Questions}
\frame
{
  \frametitle{Questions}
\begin{center}
 \includegraphics[width=125pt]{doubtux.png}
\end{center}
}

%Final slide 
{
\setbeamertemplate{background
canvas}{\href{http://www.hpckp.org}{
\hspace{-0.3cm}
\includegraphics[height=\paperheight]{hpckp_slide01.png}}}
\begin{frame}[plain]
\end{frame}
}

\end{document} 